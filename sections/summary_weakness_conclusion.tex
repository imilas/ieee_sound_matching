\section{Summary, Weaknesses, and Conclusion}
\label{sec:summary_conclusion}
\textbf{Summary:} Sound-matching is an umbrella term for the algorithmic programming of audio synthesizers, often with the goal of assisting sound designers. Here we provided a history of sound-matching, major issues in the field, and discussed the importance of ``differentiable iterative sound-matching'' as a natural extension of current literature.

The main hypothesis tested here is whether the performance of differentiable loss functions is influenced by the synthesis techniques used for sound-matching. We created iterative sound-matching experiments by combining four different loss functions with four different sound synthesis programs. We ranked the performance of the iterative sound-matching pipelines for every loss function and program, and observed that the success of the pipeline (that is, how closely the output sound matches the target sound) is program dependent. In other words, different synthesizer programs work best with different loss functions. Notably, we see that our novel use of DTW and SIMSE based differentiable loss functions (\DTWEnv{} and \SIMSESpec) can outperform what are regarded as the SOTA loss functions in 3 of 4 cases, although this is highly synthesizer dependent. 



P-Loss and MSS have frequently been used as automatic performance measures, yet their ``preference'' has rarely been compared to human rankings. We observed that automatic performance measures and manual listening tests were often in agreement, despite this, manual verification of sound-matching results remains a necessity for future works.  

\textbf{Weaknesses:} 
While we cannot prove that a universally best similarity measure does not exist, we can advocate for more creativity in the field based on our findings. Compared to previous work, we presented a more cohesive approach to iterative sound-matching which utilizes a variety of loss functions and synthesis methods. However, there are many other methods of synthesis and sound-similarity that can be combined in practically infinite ways. Due to this large search space, we set arbitrary parameters for the various signal processing functions. We used bare-bones versions of STFT and JTFS with fixed parameters. We did not test complex synthesizers using parallel and sequential DSP functions. Arbitrary hyperparameters such as learning rate and max number of iterations were selected for the DL pipeline, and only the RMSProp optimizer was tested. 

 Like the majority of previous works, this work utilizes in-domain sounds; that is, the target sound is made by the synthesizer, and the parameters are already known. This simplifies the issues of measuring sound-similarity, but it is not a realistic scenario for practical sound-matching. This problem is left for future work, discussed in the next section.

\label{sec:future}
\textbf{Future Work: } The problem of periodic loss-landscapes, noted by many previous works, remains unaddressed~\cite{turian2020sorry,vahidi2023mesostructures,uzrad2024diffmoog,bruford2024synthesizer}. Perhaps this problem emerges due to the periodic nature of sound, which requires better loss landscape navigation methods. An optimizer that is more aware of fluctuations in the gradients would perhaps lead to better solutions than simple gradient descent. Viewing the sound-matching problem as ``the navigation of an agent from an arbitrary point in a gradient field to a target'' closely resembles many classical problems in the field of reinforcement learning (RL)~\cite{sutton2018reinforcement}. Naturally, the application of RL and other heuristic search techniques to the problem of iterative sound-matching would be an important contribution.

Contemporary works often involve the application of domain specific and computationally expensive loss functions~\cite{han2023perceptual,uzrad2024diffmoog}, use of large neural networks~\cite{hershey2017cnn,cramer2019look}, or complex ensemble methods~\cite{turian2022hear}. Such models are useful but intractable; furthermore, training them requires the definition of simpler loss functions, which emphasizes the need for further development of differentiable and expressive loss function implementations. 

Like nearly all previous work in sound-matching, the main measure of success here was the accurate \textit{replication} of sounds (or synth parameters) rather than imitation of sounds outside of the synthesizer's domain. Replication of arbitrary features of sounds is an important component of sound-design, though it is much harder to define or measure. Future works could explore loss functions which only measure certain characteristics of sound (such as \DTWEnv), and whether they can pave the way for better imitation in sound-matching.
