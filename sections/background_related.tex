\section{Background and Related Works}

\todo{Sound-matching is an amalgamation of various fields and  requires knowledge of digital signal processing, audio representation, and heuristic search. Here we introduce the various fields and ideas behind sound-matching, before we use these ideas in a historical framing of past sound-matching works.}

\subsection{Digital Audio Synthesis}
\label{sec:dsp}
A digital audio synthesizer is any software used for the creation and manipulation of digital audio. Digital synthesizers use signal processing chains with a variety of digital signal processing (\gls{DSP}) functions to create sounds. These functions are often parametric, and the set of parameters given to the synthesizer's chain of DSP functions is called a synthesizer program.

The simplest form of DSP function is a sinusoidal tone. For example, consider:
\[ x(n) = sin( 2\pi n T)\]

where $T$ represents time in seconds, and $n$ is the frequency parameter. If $n$ is 1, then $x(n)$ would be a waveform with a frequency of 1 hertz, meaning that the value of $x(n)$ oscillates to the same point once every second. Waveform generators are called oscillators for this reason. 

Since the advent of digital signal processing in the 1960s~\cite{stranneby2004digital}, a wide variety of parametric audio generation functions have been proposed~\cite{lyons1997understanding,russ1999sound,shier2020spiegelib}. This includes oscillators, filters, equalizers, and envelopes, which can be used sequentially or in parallel~\cite{lyons1997understanding,russ1999sound}. Sound design with a synthesizer is done by modifying the parameters of the functions until reaching a desired output~\cite{roads1996computer,pinch2004analog}.

Particularly in earlier works, frequency/amplitude modulators (\gls{FM}/\gls{AM}) has been the synthesizer of choice in sound-matching~\cite{horner1993machine,mitchell2007evolutionary,vahidi2023mesostructures}. This synthesis method is simple to implement, yet very expressive~\cite{chowning1973synthesis}. A common implementation is the use of one carrier, one modulator, and the corresponding envelopes:
\[ f(t) = I_c(t) \cos(\omega_c t + I_m(t) \cos(\omega_m t))
\]
Where $f(t)$ is the output signal, $I_c(t)$ is the carrier amplitude, $\omega_c$ is the carrier frequency, and $I_m(t)$ and $\omega_m$ are the envelope and frequency for the modulator~\cite{justice1979analytic}. 

Compared to the number of possible approaches to sound synthesis, a relatively small subset of synthesis methods have been analyzed in \textit{isolation} for sound-matching. Notable methods here are additive and/or subtractive synthesizers~\cite{engel2020ddsp,masuda2023improving,salimi2020make} and physical modeling~\cite{riionheimo2003parameter,han2024learning}.

\highlight{
What we mean here by \textit{isolation} is that the effect of the synthesizer's parameters (how they change the output sound and experiment outcome) is easily tractable. Recent computational advancements have allowed the use of complex Virtual Studio Technology (VST)\cite{steinberg1996vst} synthesizers in sound-matching~\cite{yee2018automatic,esling2019flow}, however, the trade-off here is the obfuscation of the interactions between the synthesizer functions, the loss, and the final outcome, among other distinctions which we will discuss in this section. 
}

An important recent development in digital audio synthesis is the increase of interest in differentiable DSP (\gls{DDSP})~\cite{engel2020ddsp}. DDSP is a general term for applications that combine machine learning techniques such as gradient descent~\cite{goodfellow2016deep,boyd2004convex} with DSP. Implementing complex DSP functions in a differentiable manner can be challenging, and effective differentiable audio similarity measures require careful mathematical expression of the desired attributes of sound. These issues have likely contributed to the limited exploration of this domain~\cite{masuda2021soundmatch,vahidi2023mesostructures,uzrad2024diffmoog}. 


\subsection{Sound Representation and Loss Functions}
\label{sec:loss_funcs}
A digital sound (or an audio signal) is a series of numbers~\cite{smith1991viewpoints,smith2007mathematics}. To compare two digital sounds, the two corresponding series are passed to a function that measures their similarity. Two signals can sound identical to our ears, without having any values in common~\cite{moore2012introduction}. This necessitates the use of proxy representations (or feature extractors) when comparing sounds automatically. Similarity between the target sound and the synthesizer output is then measured by some form of subtraction and summation of the proxy representations.

In sound-matching, particularly in a Deep Learning (\gls{DL}) context~\cite{goodfellow2016deep}, the similarity function can also be called a \textit{loss} function, where the emphasis is on measurement and reduction of the distance between target and output. It is important to note that there is a close relationship between the loss function $L$ and the sound representation function $\phi$. $L$ is the result of a distance measure $d$ applied to the features extracted by $\phi$. 

\[
L(\theta, t) = d\langle\phi(t),\phi(x)\rangle
\]

\noindent

A proxy representation is the output of the function \( \phi \), which can be thought of as a feature extraction function that maps the sounds \( t \) and \( x \) to their respective representations. 
The proportionality or distance metric $d$ has typically been the L1 or L2 distance~\cite{turian2020sorry,richard2025model}, with L1 being calculated as the mean of the absolute difference between every point in the proxy representation:
\[
L(\theta, t) = \left\| \phi(t) - \phi(x) \right\|_1
\]

Here we discuss four methods of audio representation and the corresponding loss functions. We also include a technical justification for our novel methods of defining sound similarity using the Scale-Invariant Spectrogram Error (SIMSE) and Dynamic Time Warping (DTW) techniques. 

\subsubsection{Parameter Loss}
A common measure of similarity in sound-matching is the distance between synthesizer parameter sets, referred to as ``P-Loss"~\cite{han2023perceptual}. Typically, for the implementation of P-Loss the parameter sets are treated as vectors in space, and L1 or L2 distance is applied. There are two major limitations to this approach: First, the target and output sound must be made by the same synthesizer; otherwise the parameter sets cannot be compared (see Section~\ref{sec:matching_types}). Second, the relationship between synthesizer parameters and the audio output is not linear~\cite{shier2020spiegelib,han2023perceptual,esling2019flow}. 

\subsubsection{Fourier Spectrograms}
\label{sec:fourier_specs}
Fourier-based transformations such as short-time Fourier transforms (\gls{STFT}), Mel-spectrograms, and Mel-frequency cepstral coefficients have been viewed as the de facto and state-of-the-art representation of audio~\cite{beauchamp2003error,mitchell2007evolutionary,yee2018automatic}, however, there are many issues associated with their use in sound-matching~\cite{turian2020sorry,vahidi2023mesostructures,han2023perceptual,uzrad2024diffmoog}. Fourier transformations allow for the conversion of a signal from the time-domain to the frequency domain. Audio spectrograms can be generated by segmentation of a piece of audio into overlapping windows followed by the application of Fourier transforms to each window. They are costly to compute, but provide a better temporal view of changes in frequency content~\cite{muller2007dynamic,smith2007mathematics}. There are different types of spectrograms that have a basis in Fourier transformations, but the most notable and commonly used is the STFT.  

What we call \textit{Fourier-based Spectrograms} are variations on the STFT approach. For example, Mel-Spectrograms \textit{bin} frequencies on a near-logarithmic scale to better match human perception of frequencies~\cite{muller2007dynamic}. Multi-scale spectrograms (\gls{MSS}) used in recent works are a simple weighted average of multiple spectrograms with different parameters such as window size, number of frequency bins, and hop size~\cite{engel2020ddsp,vahidi2023mesostructures}; this may provide some improvements at a higher computational cost~\cite{turian2020sorry,engel2020ddsp}.

\textbf{Why use Scale-Invariant Spectrogram for audio loss}
A limitation of L1/L2 distances on spectrograms is their sensitivity to global gain: 
two sounds with identical spectral shape but different amplitude can yield large losses. 
Scale-Invariant Mean Squared Error (SIMSE) addresses this by normalizing the energy of the compared signals, thereby focusing on proportional differences in spectral envelopes. 
This can be advantageous for subtractive synthesis (used by one of our synthesizers), where filter cutoffs reshape the spectral balance which can cause changes in total energy. 
\todo{Perceptually, human listeners are often invariant to global amplitude changes, attending instead to relative spectral shapes}. 
Although SIMSE has been used in other contexts (e.g., image/audio reconstruction), its application as a differentiable spectrogram loss in iterative sound-matching is, to our knowledge, novel.



\subsubsection{Joint-Time Frequency Spectrum}
Recent works have focused on the limitations of parameter and spectral loss functions in sound-matching~\cite{vahidi2023mesostructures,uzrad2024diffmoog}, seeking to create more effective general solutions for the comparison of audio. 
Noting the aforementioned weaknesses of comparing STFT spectrograms, Vahidi \textit{et al.} proposed differentiable Joint-Time Frequency Scattering (\gls{JTFS})~\cite{anden2015joint} as an alternative to spectrogram loss in sound-matching, and showed improved performance in sound-matching with differentiable chirplet synthesizers~\cite{vahidi2023mesostructures}. JTFS is the result of the application of a 2D wavelet transformation to the time-frequency representation of a signal~\cite{anden2015joint}. 

\subsubsection{Dynamic Envelope Warping}
Dynamic Time Warping (DTW) is a method for measuring similarity between multi-dimensional time-series~\cite{rabiner1993fundamentals,muller2007dynamic,giorgino2009computing}. Given any two time-series $X = \{x_1,x_2,...,x_m\}$ and $Y = \{y_1,y_2,...,y_n\}$, we have indices $i\in\{1...m\}$ and $j\in\{1...n\}$ defining $X$ and $Y$. When the series are \textit{warped}, these indices change to expand or contract different portions of the series. To borrow the notation given by Muller~\cite{muller2007dynamic}, warped indices are a sequence $p=(p_1,...,p_L)$, where \(p_\ell = (m_\ell, n_\ell) \in [1 : m] \times [1 : n] \text{ for } \ell \in [1 : L]\), meaning that the indices for $X$ and $Y$ are reorganized under special conditions. In classical DTW, these conditions are \textit{monotonicity}, \textit{boundary matching}, and \textit{single step-size}.



\textbf{Why use DTW for audio loss?}
DTW measures the distance between the time-series \textit{after} alignments, typically using Euclidean distance. The distance between a time-series and shifted versions of itself would be 0, regardless of shift amount~\cite{tavenard.blog.dtw}. Additional rules can be imposed to keep alignments locally constrained~\cite{itakura1975minimum,sakoe1978dynamic}. DTW provides robustness to local temporal shifts, making it well suited for comparing modulated signals where the perceptual similarity lies in envelope dynamics rather than precise onset alignment. 
For example, two tremolo signals with slightly different phase are perceptually similar but would appear distant under spectrogram L1. 
By applying DTW specifically to amplitude envelopes, we align with perceptual cues of loudness modulation while maintaining differentiability. 
To our knowledge, DTW applied in this way has not been used as a loss in sound-matching (whether differentiable and iterative or not).



\subsection{Automatic Sound Matching}
\label{sec:sound_matching_definition}
The problem of sound-matching has been tackled from a variety of perspectives as it involves a number of modular parts: the choice of synthesizer $g$, the target sounds of interest, the representation function $\phi$, and the heuristic for finding the optimal $\theta$. To give a formal definition of sound-matching, we expand the definition given by recent works~\cite{vahidi2023mesostructures,han2023perceptual}. The major components are as follows: 
\begin{itemize}
    \item $g(\theta)$: Parametric audio synthesizer $g$ which takes on parameters $\theta$ 
    \item $x$: The output of $g$, given a set of parameters $\theta$ or $g(\theta) = x$ 
    \item $t$: The target sound which we want to replicate or imitate. 
    \item $\phi$: Representation function or feature extractor. $\phi$ is applied to $x$ and $t$ to facilitate their comparison by the loss function.
    \item $L$: Loss or error function. $L(\theta,t)$ is a measure of distance between $x$ and $t$. Often proportional to the subtraction of their representations, or $ \phi(x) - \phi(t)$
\end{itemize}

\highlight{It could be argued that the implementation of these components is not an optimization problem with a correct answer, but a \textit{creative} endeavor depending on the artist's needs}. Consider the hypothetical case where the goal might be to create an 8-bit~\cite{collins2007loop} version of a high quality, organic snare drum: in such a case, there is no optimal or ``correct" version of an 8-bit snare sound. 

\subsection{In-Domain versus Out-of-Domain}
\label{sec:in-domain}
The choice of domain depends on whether we want to use the same synthesizer for the target and output sounds, the scenario that is called \textit{in-domain}, or have target sounds that came from sources other than the synthesizer, or \textit{out-of-domain}. To paraphrase the description given by Masuda \textit{et al.}~\cite{masuda2021soundmatch}, if $g$, the synthesizer of choice, can accurately replicate the target sound $t$, or put differently, if $t$ itself is an output of $g$, then the sound-matching task is \textit{in-domain}. If $t$ is not an output of $g$, then the sound-matching task is \textit{out-of-domain}. In-domain tasks in general are simpler, and often there is a guarantee that there is a correct answer to the sound-matching problem, particularly if the goal is accurate replication of the sound. If the target sound is out-of-domain, replication is not guaranteed, and the goal becomes the \textit{imitation} of some aspect of sound. 

Regardless of the domain, the generation goal can be \textit{replication} or \textit{imitation} of the target sound. In replication, the goal is to make an identical copy of the target sound. Imitation is an artistic pursuit and harder to define, since the goal is to make new sounds that only retain a subset of the target's sonic features. While closely related to the in-domain versus out-of-domain problem, the choice of replication versus imitation is more dependent on how the loss and representation functions are defined. 

\subsection{Supvervised vs Direct Optimization}
\label{sec:optimization}

The choice of \textit{heuristics} is yet another important attribute sound-matching. The goal of sound-matching is to find the optimal parameters $\theta^*$ that minimize the loss between the synthesizer output and the target sound. 
\[
\theta^* = \arg\min_{\theta} L(\theta,t)
\]

The heuristics (i.e., how $\theta^*$ is approximated) used in past works can be broadly split into the two categories of \textit{direct optimization} and \textit{supervised}  (or inference) methods. Direct optimization refers to the iterative generation of a sound output, measurement of similarity between target and output, and application of updates to the parameters to maximize similarity (or minimize loss)~\cite{horner1993machine,mitchell2007evolutionary,yee2018automatic,vahidi2023mesostructures}; while supervised methods use large datasets of synthesizer sounds and corresponding parameters to learn the generation objective, commonly with the use of DNNs~\cite{engel2020ddsp,salimi2020make,yee2018automatic,esling2019flow}. These models often make their parameter predictions in a single step (or 1-shot). 

Genetic algorithms (\gls{GA})~\cite{holland1992genetic} have been the earliest and most common heuristic for direct sound-matching~\cite{horner1993machine,mitchell2007evolutionary,yee2018automatic}. These algorithms start with arbitrary parameter sets that can be treated as an evolving population where the genomes are the parameter values. The most fit members of the group are the parameters which perform the best in the loss function, and create a new generation of parameters via mutation (random change in subset of parameters) and crossovers (combination of parameter sets); this process repeats until the goal or a maximum number of generations is reached. Rather than using random mutations, differentiable approaches allow goal-oriented updates to the synthesizer parameters (the goal being the minimization of loss), but with some drawbacks: other than requiring more computation power, differentiable functions require careful implementation of operations that are continuous and numerically stable; this makes the implementation of signal processing functions quite difficult, contributing to the scarcity of works in this domain.

\subsection{Historical Framing of Sound-Matching}
Given the different attributes discussed thus far, Table~\ref{tab:summary} contains an overview of important relevant literature. This leads us to a more detailed historical analysis of past works in sound-matching.

Perhaps the earliest foundational work in sound-matching is the analytical approach by Justice~\cite{justice1979analytic} toward the decomposition and recreation of sounds using the simple FM synthesizer described in Section~\ref{sec:dsp}. The sound-matching works that followed used other heuristics for finding the synthesizer parameters, yet the simple FM synthesizer structure remained unchanged. Horner \textit{et al.}~\cite{horner1993machine} used GAs for re-synthesis of sounds with an FM synthesizer using one modulator and one carrier oscillator, and the McAulay-Quatieri method~\cite{mcaulay1986speech} for measuring loss. 


Introduction of more complex models of synthesis such as wavetables~\cite{horner2003auto} and physical modeling~\cite{riionheimo2003parameter} rendered interesting results, but led to further questions about the nature and goals of sound-matching. Mitchell and Creasy noted the difficulty in distinguishing between ``inefficiency of the optimization engine'' and ``synthesizer limitations'' as the cause of failure or success in sound-matching~\cite{mitchell2007evolutionary}. They offered a \textit{contrived methodology} for finding the best evolutionary method for sound-matching using their FM synthesizer. The methodology postulates that the best search heuristic for in-domain search (where an exact target exists, and a wide range of targets can be produced by sampling from different points in the parameter space) would also be the best-performing heuristic for out-of-domain search on a dataset of muted trumpet tones~\cite{opolko1989mcgill}; however, testing was not extensive and yielded some contradictory results. As they noted, modifications to the synthesizer, loss function, and sound domain would lead to a problem with an entirely new search space~\cite{mitchell2007evolutionary}. 

Recent years have seen more works in sound-matching using supervised machine learning techniques. In 2018, Yee-King \textit{et al.} rendered 60,000 audio-parameter pairs from the \textit{Dexed} \gls{VST} synthesizer~\footnote{https://asb2m10.github.io/dexed/}, and showed that NNs trained on this dataset can outperform GA and hill-climber (\gls{HC})~\cite{hoffmann2000heuristic} methods in rendering speed, with slight improvements in MFCC error (used as an objective performance test). The speed improvement appears trivial, considering the iterative nature of GAs when compared to offline training of supervised models. For GAs and HC optimizer, MFCCs were used as a measure of performance as well as a loss function. For training the networks, P-Loss was used, since differentiable MFCCs were not possible in their pipeline. Importantly, informal hearing tests revealed that the performance of even the best NN model was unsatisfactory, possibly due to the complex nature of the synthesizer, which features 155 parameters.

Masuda \textit{et al.} also applied supervised learning to synthesizer parameter estimation~\cite{masuda2021soundmatch}. Their work highlights the issue of non-linearity in parameter-to-synthesizer outputs and out-of-domain search. This work uses a differentiable subtractive synthesizer (two oscillators and an LP filter). A NN model was pre-trained using an in-domain dataset of randomly selected parameters and P-Loss. After training, the model was fine-tuned using 20,000 out-of-domain sounds from the NSynth dataset~\cite{engel2017neural} and multi-scale spectrogram loss~\cite{engel2020ddsp}. This approach proved more effective---i.e, achieved lower multi-level spectral difference in out-of-domain tests---than baseline models, which were either not exposed to out-of-domain sounds or trained exclusively with P-Loss. Subjective hearing tests were conducted, showing a preference for the fine-tuned model~\cite{masuda2021soundmatch}. Masuda \textit{et al.} later extended this work with semi-supervised learning, highlighting significant gaps in in-domain and out-of-domain performance~\cite{masuda2023improving}.

Rather than focusing on a particular implementation, Shier \textit{et al.} presented Spieglib, a library for implementation of sound-matching pipelines~\cite{shier2020spiegelib}. This library provides different choices for DNNs, GAs, synthesizers, and feature extractors. Shier \textit{et al.} presented an experiment with a similar setup to Yee-King \textit{et al.}~\cite{yee2018automatic}, however, they found a genetic algorithm to be the best performing.

Differentiable loss functions that use spectrogram differences can be computationally expensive. To mitigate this, Han \textit{et al.}~\cite{han2023perceptual} introduced ``perceptual-neural-physical loss'' (PNP). PNP is an approximation of loss functions; specifically, a loss function that uses the L2 norm of the difference between the features of two sounds, or $||\phi(t) - \phi(x)||^2_2$, where $\phi$ could be a spectrogram or JTFS function. PNP loss functions are fast and differentiable, but require training and parameter estimation. A Riemannian metric M needs to be calculated for the minimization of locally linear approximation of the ``true" spectral loss function~\cite{han2023perceptual}. 
\[
\|\phi(t) - \phi(x)\|_2^2 = \langle \tilde{\theta} - \theta | \mathbf{M}(\theta) | \tilde{\theta} - \theta \rangle + O(\|\tilde{\theta} - \theta\|_2^3). \tag{4}
\]

During the training phase, this metric is calculated alongside the neural parameter estimator. After training, PNP can be used as a fast approximation of computationally demanding loss functions such as JTFS~\cite{han2023perceptual,han2024learning}, paired with FM synthesizers and a differentiable physical model of a drum head~\cite{smith2010physical}.

A neural approximation for another part of the supervised sound-matching chain---this time the synthesizer---was proposed by Barkan \textit{et al.}~\cite{barkan2023inversynthII}. As they noted, without a differentiable synthesizer, ``model-based'' (or what we call supervised) approaches cannot directly compare $x_o$ to $x_t$, often opting for P-Loss, which may not correctly map the parameters of sound to the output audio~\cite{esling2019flow,han2023perceptual,masuda2023improving}. Given a synthesizer and a large dataset of sounds and corresponding parameters, a model which maps sounds to parameters, and another model which approximates the synthesizer can be trained, allowing the implementation of a loss function which combines P-Loss with STFT differences. This ``Inversynth II'' (IS2) approach yielded significant improvements to previous works which did not use the STFT approximations for loss~\cite{esling2019flow,barkan2019inversynth}.

Barkan \textit{et al.} attemped to  improve the IS2 model with Inference-Time
Finetuning (ITF). For ITF, the synthesizer approximation is frozen and the encoder is iteratively finetuned for the sample sample using gradient descent in order to minimize P-Loss, stopping at convergence or when maximum number of iterations has reached~\cite{barkan2023inversynthII}. Barkan \textit{et al.} report that perhaps due to the discrepancy between the true and approximated synthesizer or overfitting, the application of ITF to IS2 generally worsened the spectral and manual performance metrics.


Uzrad \textit{et al.}~\cite{uzrad2024diffmoog} took another unique approach to sound-matching: using a differentiable \textit{synthesis chain} of DSP generators and effects and a loss function that combines P-Loss with a \textit{signal-chain loss}. The synthesizer is a customizable chain of effects, which feed one output as input to the next step of the chain; signal-chain loss compares the parameter and output difference at every output step in the chain~\cite{uzrad2024diffmoog}. Possible chain functionalities are FM/AM, Low-Frequency Oscillators (\gls{LFO}), filters, and envelopes. Like the results shown by Masuda \textit{et al.}~\cite{masuda2021soundmatch}, better out-of-domain results were achieved when pre-trained on in-domain data and fine-tuned using out-of-domain NSynth data~\cite{engel2017neural}.


Some recent works have opted for the use of audio embeddings as a similarity metric. In a recent work, Cherep \textit{et al.} used latent representations from the CLAP model~\cite{wu2023large} along with a differentiable synthesizer~\cite{synthhaxcherep2023} to create creative interpretations of sound effects. In their approach, a desired sound-effect is described in text and embedded using clap, and gradient free optimizer~\cite{evosax2022github} iteratively updates the synthesizer's parameters to minimize difference in the text and output sound embeddings. Based on manual hearing tests, this approach did not produce the ``correct'' sounds more frequently than previous works~\cite{kreuk2022audiogen}, however, it did yield better scores for ``artistic-interpretation'' (or what we refer to as imitation). 


% masuda and esling both noted out-of-domain drop in performance
% yee king noted performance drop in manual surveys 
%p-loss bad : masuda-improving, han2023, esling-flowsynth
% manual performance measure: barkan, masuda2021soundmatch, yee-king 

\subsection{What is Lacking In the Field}
\label{sec:lacking}
%   % First, we need to clearly define the characteristics we are looking for with non-generic loss functions, otherwise the chances of finding the sound(s) of interest is slim. Second, we need to move beyond direct replication, and focus on imitating target sounds that are outside the synthesizers capabilities. Third, more complex methods of synthesis need to be explored for iterative sound-matching. Fourth, we need to find optimization methods that can effectively navigate the the sinusoidal nature of loss function landscapes. 

% Based on the literature reviewed, we find several gaps in past research that makes the application of sound-matching for sound-designers difficult. Thus far, replication of in-domain sounds has been the main subject of research. With in-domain replication, we already have the target sound \textit{and} the program that makes it. This simplifies the problem of measuring performance and is the reason why this type of sound-matching research is by far the most common. However, practical application of sound-matching requires answers to much harder problems. Sound-matching would be an interesting creative tool if we could use it to either:
% \begin{enumerate}
%  \item \textbf{Replicate out-of-domain sounds}: this would give us the parameters that could closely approximate a target sound, and we can modify the sound further by modulating the synthesizer. 
% \item \textbf{Imitate in-domain or out-of-domain sounds}: This would allow for applying the characteristics of one sound to another. 
% \end{enumerate}

% We need to be aware of the ``open set recognition'' (\gls{OSR}) problem here \cite{mundt2019open,gers2000learning}. OSR is a well documented issue in DL that manifests when the set or category that the network is supposed to reject or accept cannot be explained via examples. With imitation or replication of out-of-domain sounds, we can expect issues related to OSR to occur, since target and output sounds would be from different domains~\cite{salimi2021percussive}. 

% Perhaps one method of addressing the OSR problem is the use of representations (or feature extraction) methods that reduce sounds to a feature vector in the same domain. For example, a representation that only looks at the envelope of sounds is far less likely to be subject to the OSR problem. \textcolor{highlight}{We highlight the lack of non-generic loss functions as a major weakness in previous works and an interesting area to explore}. 

% Another weakness \textcolor{highlight}{is that the synthesis methods in past works are not representative of the common methods used by sound designers and modern synthesizers}. Simple FM/AM synthesizers have generally been the go-to methods of sound-matching. We believe that general statements regarding the effectiveness of an approach to iterative sound-matching (particularly in regards to a loss function) can only be reasonably made in the context of the subset of the tested methods of synthesis.
% \textcolor{highlight}{The simplicity of parameter optimization approaches} is another area of weakness in past works. Classic genetic algorithms update the parameters randomly, while in recent differentiable settings, simple gradient descent methods are used. Past works have provided issues which arise in differentiable parameter estimation due to periodic loss functions. Perhaps the use of reinforcement learning (\gls{RL}), or other algorithms that continually learn from their environment, could help with this problem.

% \subsection{Defining The Major Issues}
% \label{sec:addressing_problems}

Having looked at the current available literature, we note four major areas of weakness in sound-matching. We target the first two issues in this work, while the latter two issues are left for future work. 
\begin{enumerate}
    \item \LossSelect: Is there a \textit{best performing} loss function, or is the selection of the loss function dependent on factors such as the sound domain, synthesizer capability and the desired characteristics of the output? 
    \item \SynthSelect: There is a lack of diversity in the synthesis methods used in sound-matching. The use of different DSP functions for iterative sound-matching and their effects on the outcome or the interactions with the loss function have not been tested.
    \item \PeriodicLoss: The sinusoidal nature of loss function landscapes is a problem that appears frequently. In differentiable settings, this problem causes gradient descent updates to local minima. 
    \item \OutDomain: Sound-matching with out-of-domain sounds is an unexplored area, yet a necessary one for practical applications for sound designers.
\end{enumerate}


